\section{Changes since version 0.91}

The following changes are incorporated:
\begin{itemize}
\item Two multi-threading engines are now available for analyzing
multi-threaded programs. Currently, documentation for using them
is only available from
\url{http://www.cs.tau.ac.il/~yahave/3vmc.htm}.

%\item Many-sorted logics is supported with predicates of arbitrary
%order, see \secref{Predicates}. Type checking will be conducted.

\item A properties mechanism is added. This is described in
\ref{Se:PropertyFiles}.

\item The command-line options -b2 and -rotate have been moved to
property file.

\item The -significant option is no longer supported. This was
done in order to offer better performance by adding different
implementations for three-valued structures.

\item The -dump option is no longer supported.

\item The command-line option -join has been renamed to -save.

\item The command-line option -single has been replaced with the
-join option, which also includes a partial join (see the
command-line options section for more details).

\item The command-line option -action includes blur as a mandatory
operation, which is applied last.

\item The following command-line options are new :
%%-mode, -backward,
-props, -tvs, -dot, -D, -terse,  and -nowarnings. Consult the
command-line options sections before using them.

\item It is now possible to specify empty predicate update
sections.

\item Nullary predicates can be presented as either diamonds or
listed in a box. Use the tvla.dot.nullaryStyle property to choose
the desired presentation.

%\item Properties are used to control various the graphical
%representation of predicates, which are no longer part of the TVP
%specification. Use the tvla.dot.boxedPredicates to specify which
%predicates should be displayed in a box and the
%tvla.dot.displayedKleeneValues property to specify which Kleene
%values should be displayed for predicates.
\end{itemize}
